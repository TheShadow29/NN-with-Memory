\documentclass{article}
\usepackage[a4paper, tmargin=1in, bmargin=1in]{geometry}
\usepackage[utf8]{inputenc}
\usepackage{graphicx}
\usepackage{parskip}
\usepackage{pdflscape}
\usepackage{listings}
\usepackage{hyperref}
\usepackage{float}
\usepackage{caption}
\usepackage{subcaption}
\usepackage{amssymb}

% \usepackage{siunitx}
% \sisetup{round-mode=places,round-precision=2}

\newcommand{\ra}{$\rightarrow$}


\title{CS 725 Machine Learning Project Proposal}
\author{
  Arka Sadhu 140070011\\
  Parth Kothari 14D070019\\
  Shray Sibal 14D070017\\
  Varun Bhatt 140260004\\}
\date{\today}

\begin{document}
\maketitle

% \tableofcontents
% \newpage
\section{Topic}
Deep Reinforcement Learning using Neural Networks with Memory

\section{Project description}
The project involves using a neural network with memory to be able to perform well on games which require the agent to remember things from the past.

\section{Approaches}
The main idea is to use traditional reinforcement learning methods like Deep Q network (DQN) supplemented with memory. There will be a few CNN layers in the beginning and a network for Q learning in the end. In between, we will explore different methods of adding a memory layer. Initially, we aim to implement the architectures given in the below papers and see how it performs in the game that we have chosen. Later, we may make improvements to it if possible.

\section{Research papers}
\begin{itemize}
\item Control of Memory, Active Perception, and Action in Minecraft \newline
  [\url{http://proceedings.mlr.press/v48/oh16.pdf}]
\item Neural Map: Structured Memory For Deep Reinforcement Learning \newline
  [\url{https://arxiv.org/pdf/1702.08360.pdf}]
\item Neural Turing Machine \newline
  [\url{https://arxiv.org/pdf/1410.5401.pdf}]
\end{itemize}


\section{Data sets}
In our case it would be a game instead of dataset. The game should be such that memory would be useful in playing well which means, ideally, the game should be a partial information game and an observation in past will be useful in making a decision at present. Strategy games with fog of war are ideal but they have a huge action space which may make it too hard to train the agent with the hardware resources that we have. Hence, as of now, one of the games we are considering is Montezuma Revenge Atari game. Other option could be creating custom tasks like it is done in “Control of Memory, Active Perception, and Action in Minecraft”

\section{Project Details}
\subsection{Implementation}
The implementation of the project will be in stages. First we will reproduce the memory architecture given in the first paper [Control of Memory, Active Perception, and Action in Minecraft]. The code for that paper is available but it is in lua. We want to convert it to python and make it work with OpenAI gym so that we can test the architecture with the games available in it. Then, we will look into other memory architectures.

\subsection{Programming Languages}
The implementation will be done in Python. We have decided to use Tensorflow for coding the neural network and memory architectures. Also we would use OpenAI gym for games (other environments may be used later if different types of games are required)

\subsection{Contributions and Implementation for each member}
We haven't yet decided who would do what. We will break up the task so that 2 members are responsible for one task. Initial plan is that 2 people work on converting the code (as described in implementation) while 2 work on reading about or implementing other memory architectures

\section{Member Details}
\begin{itemize}
\item Arka Sadhu 140070011
\item Parth Kothari 14D070019
\item Shray Sibal 14D070017
\item Varun Bhatt 140260004
\end{itemize}
All are from Electrical Engineering Department.

\end{document}